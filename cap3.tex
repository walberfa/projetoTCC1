\chapter{Metodologia}\label{CAP3}

O desenvolvimento deste trabalho deve ocorrer em cinco etapas: escolha e uso da ferramenta de simulação de sistemas digitais, implementação dos códigos, escolha da ferramenta de injeção de falhar, experimentação e montagem dos quadros comparativos.

\begin{figure}[ht]
\centering
\includegraphics[scale=0.8]{figures/diagrama.png}
\caption{Etapas para o desenvolvimento do trabalho}
\label{fig:diagrama}
\end{figure}

Etapa 1: Escolha e uso da ferramenta de simulação de sistemas digitais.

Nesta primeira etapa, serão estudadas algumas ferramentas de simulação para sistemas digitais com o objetivo de escolher uma que possa contemplar a todos os parâmetros que se pretende utilizar.

Etapa 2: Implementação dos códigos.

Neste etapa, será feita dois de tipos de implementação dos códigos a serem comparados: a da matemática por trás dos códigos, que será feita em Python e/ou Matlab, e a implementação em VHDL, na ferramenta de simulação escolhida na etapa anterior. Neste segundo tipo de implementação já poderão ser calculados alguns parâmetros, como potência e area ocupada.

Etapa 3: Escolha da ferramenta de injeção de falhas.

Serão analisadas algumas ferramentas de injeção de falhas disponíveis do mercado, assim como as utilizadas pelos autores dos trabalhos estudados. Definida qual a ferramenta, podemos ir para a próxima etapa.

Etapa 4: Experimentos.

A etapa de experimentação é bastante importante, pois é onde podemos obter os dados das taxas de correção e detecção de falhas. A ferramenta escolhida, irá injetar milhares de falhas pseudo-aleatórias nos sistemas de memória, afim de que os códigos corretores de erro encontrem e corrijam estas falhas.

Etapa 5: Montagem dos quadros comparativos.

Esta é a última etapa do trabalho, quando já dispormos de todos os dados necessários para a comparação entre os códigos. Serão montados três quadros: desempenho, implementação e abordagem.
