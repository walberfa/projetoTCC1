
[1]TANENBAUM, Andrew S. Organização Estruturada de Computadores. 6 edição. São Paulo: Pearson, 2013.

[2]CHUGG, A. M.; MOUTRIE, M. J.; JONES, R. Broadening of the variance of the number of upsets in a read-cycle by MBUs. IEEE Transactions on Nuclear Science, v. 51, n. 6, p. 3701-3707, 2004.

[3]SLAYMAN, Charles W. Cache and memory error detection, correction, and reduction techniques for terrestrial servers and workstations. IEEE Transactions on Device and Materials Reliability, v. 5, n. 3, p. 397-404, 2005.

[4]VARGHESE, Bibin et al. Multiple bit error correction for high data rate aerospace applications. In: 2013 IEEE Conference on Information and Communication Technologies. IEEE, 2013. p. 1086-1090.

[5] ARGYRIDES, Costas; ZARANDI, Hamid R.; PRADHAN, Dhiraj K. Matrix codes: Multiple bit upsets tolerant method for SRAM memories. In: 22nd IEEE International Symposium on Defect and Fault-Tolerance in VLSI Systems (DFT 2007). IEEE, 2007. p. 340-348.

[6]CASTRO, Helano de S. et al. A correction code for multiple cells upsets in memory devices for space applications. In: 2016 14th IEEE International New Circuits and Systems Conference (NEWCAS). IEEE, 2016. p. 1-4.

[7]SILVA, Felipe et al. An efficient, low-cost ECC approach for critical-application memories. In: Proceedings of the 30th Symposium on Integrated Circuits and Systems Design: Chip on the Sands. 2017. p. 198-203.

[8]MAGALHÃES, Philippe; ALCÂNTARA, Otávio; SILVEIRA, Jarbas. PHICC: an error correction code for memory devices. In: 2019 32nd Symposium on Integrated Circuits and Systems Design (SBCCI). IEEE, 2019. p. 1-6.

[9]SATOH, S.; TOSAKA, Y.; WENDER, S. A. Geometric effect of multiple-bit soft errors induced by cosmic ray neutrons on DRAM's. IEEE Electron Device Letters, v. 21, n. 6, p. 310-312, 2000.

[10]IBE, Eishi et al. Impact of scaling on neutron-induced soft error in SRAMs from a 250 nm to a 22 nm design rule. IEEE Transactions on Electron Devices, v. 57, n. 7, p. 1527-1538, 2010.

[11]SILVA, Felipe et al. Evaluation of multiple bit upset tolerant codes for NoCs buffering. In: 2017 IEEE 8th Latin American Symposium on Circuits and Systems (LASCAS). IEEE, 2017. p. 1-4.

[12]LAWRENCE, Reed K.; KELLY, Andrew T. Single event effect induced multiple-cell upsets in a commercial 90 nm CMOS digital technology. IEEE transactions on nuclear science, v. 55, n. 6, p. 3367-3374, 2008.

[13]SANCHEZ-MACIAN, Alfonso; REVIRIEGO, Pedro; MAESTRO, Juan Antonio. Hamming SEC-DAED and extended hamming SEC-DED-TAED codes through selective shortening and bit placement. IEEE Transactions on Device and Materials Reliability, v. 14, n. 1, p. 574-576, 2012.